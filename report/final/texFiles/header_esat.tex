% filename: header_esat.tex
% auteur: Dirk Van Hertem (dirk.vanhertem@ieee.org)
% deze  file zorgt dat de juiste  opties gezet zijn. Een  aantal opties staan niet  default, maar ik
% raad je aan ze eens te bekijken, en indien gewenst te gebruiken.


\tightlists % weinig extra witruimte bij opsommingen (memoir optie)

\usepackage[english,dutch]{babel} % nederlandse tekstopmaak % english
                                % toegevoegd in versie 0.4

\usepackage{graphicx} % Gebruik figuren
\graphicspath{{figuren/}}
\usepackage{tabularx} % maak een tabel met een bepaalde breedte (gebruikt in symbolenlijst.tex)
\usepackage{microtype} % toegevoegd in versie 0.4
\usepackage{ifthen}% toegevoegd in versie 0.5
\usepackage{calc}% nieuw in 0.6b om te kunnen rekenen in deze header
% \newboolean{fourierfont} % nieuw in versie 0.6b, op aanraden van Benoit Willems, naar 2009 toe zou ik
%                          % fourier  standaard willen zetten, maar  voorlopig nog niet  om geen grote
%                          %   veranderingen   te   krijgen    vlak   voor   de   deadline.   Vervangt
% % \usepackage{fourier} in commentaar ==> werkt nog niet, voor de volgende keer...
% \setboolean{fourierfont}{true}

% %\ifthenelse{\boolean{fourierfont}}{%

%%%%%%%%%%%%%%%%%%%%%
%%% Als je fourier wilt, uncomment de volgende regio:
%%%%%%%%%%%%%%%%%%%%%
%\makeatletter
%\let\my@@font@warning\@font@warning
%\let\@font@warning\@font@info
%\makeatother%
%\usepackage{fourier}%
%\makeatletter
%\let\@font@warning\my@@font@warning
%\makeatother

%%%%%%%%%%%%%%%%%%%%%
% inputenc
%%%%%%%%%%%%%%%%%%%%
% inputenc is afhankelijk van uw systeem (en os). Waarschijnlijk maakt
% een van de volgende \usepackage[]{inputenc} dat alles goed
% opgeslagen kan worden. Waarschijnlijk heb je latin1 nodig. \'e werkt altijd...

%\usepackage[latin1]{inputenc} % zorgt dat je ook � � en dergelijke kan ingeven
%\usepackage[utf8]{inputenc} % zorgt dat je ook � � en dergelijke kan ingeven

%%%%%%%%%%%%%%%%%%%%
% andere mogelijk nuttige pakketten
%%%%%%%%%%%%%%%%%%%%

% \usepackage{subfig}   % Gebruik subfiguren
% \usepackage{color}    % Gebruik kleuren
% \usepackage{icomma}   % zet de spati�ring goed bij kommagetallen (Europeese notatie)
% \usepackage{amssymb,amsmath,amsfonts} % extra wiskundige symbolen en
% constructies
% \usepackage{eurosym}  % \euro geeft het euroteken
% \usepackage{cite}     % verbetert de weergave van citaties 
% \usepackage{listings} % om code toe te voegen in uw document
% \usepackage{booktabs} % toegevoegd in version 0.4


% De volgende opties bepalen hoe figuren op een blad geplaatst moeten worden. Lees het volgende document voor meer informatie: http://www.ctan.org/tex-archive/info/epslatex.pdf

\setcounter{topnumber}{4}
\setcounter{bottomnumber}{4}
\setcounter{totalnumber}{10}
\renewcommand{\textfraction}{0.15}
\renewcommand{\topfraction}{0.85}
\renewcommand{\bottomfraction}{0.70}
\renewcommand{\floatpagefraction}{0.66}

% Opmerking: het volgende  stuk definieert de pagina opmaak. Indien de  keuze energie genomen wordt,
% dan wordt een opmaak met anderhalve regelafstand  genomen. Ofdat dit zinvol is, dat laat ik in het
% midden, de mensen die twijfelen lezen best de volgende url:
% http://www.tex.ac.uk/cgi-bin/texfaq2html?label=linespace. 
% De opmaak van energie heeft ook iets minder witruimte aan de randen. 
% Veranderen van opmaak kan simpelweg door de  waarde van \energie (zie regel na deze commentaar) de
% waarde "true" of "false" te geven.
\newboolean{energie}
\setboolean{energie}{false}

\ifthenelse{\boolean{energie}}% versie 0.6
{% deze optie is ook versie 0.6
% zet de pagina op (zie memman.pdf p 66)
\settrimmedsize{297mm}{210mm}{*}%
\setlength{\trimtop}{0pt}%
\setlength{\trimedge}{\stockwidth}%
\addtolength{\trimedge}{-\paperwidth}%
\settypeblocksize{654pt}{455pt}{*}% Van 634 naar 694 pt in versie 0.7
\setulmarginsandblock{2cm+\onelineskip+1.5\onelineskip}{2.65cm}{*}% Aanpassing 0.7
\setlrmargins{*}{*}{0.9}% verhouding buitenkantmarge/rugmarge. Verzet
% van 1.5 naar 0.9 vanaf versie 0.4
\setmarginnotes{17pt}{51pt}{\onelineskip}%
\setheadfoot{\onelineskip}{2\onelineskip}%
\setheaderspaces{*}{1.5\onelineskip}{*}% Aangepast versie 0.7
%
%zorg ervoor dat TOC niet te breed wordt:
%\setpnumwidth{2.5em}
%\setrmarg{3em} 
%
\checkandfixthelayout%
\OnehalfSpacing%
}%
{%
% zet de pagina op (zie memman.pdf p 66)
\settrimmedsize{297mm}{210mm}{*}%
\setlength{\trimtop}{0pt}%
\setlength{\trimedge}{\stockwidth}%
\addtolength{\trimedge}{-\paperwidth}%
\settypeblocksize{634pt}{448.13pt}{*}%
\setulmargins{4cm}{*}{*}%
\setlrmargins{*}{*}{0.9}% verhouding buitenkantmarge/rugmarge. Verzet
% van 1.5 naar 0.9 vanaf versie 0.4
\setmarginnotes{17pt}{51pt}{\onelineskip}%
\setheadfoot{\onelineskip}{2\onelineskip}%
\setheaderspaces{*}{2\onelineskip}{*}%
%
%zorg ervoor dat TOC niet te breed wordt:
%\setpnumwidth{2.5em}
%\setrmarg{3em} 
%
\checkandfixthelayout%
}
\setlength{\parindent}{0pt}
\setlength{\parskip}{0.5\baselineskip}
\setlength{\cftbeforechapterskip}{0pt}

% Voor de nummeringsdiepte te veranderen: met level 4 nummer je ook subsecties
\setcounter{secnumdepth}{4} % default gemaakt in versie 0.4
%\setcounter{tocdepth}{4}
\usepackage[hypertexnames=false]{hyperref} % geeft links zodat je een
                                % interactief document krijgt (zeer
                                % interessant als je het ooit online
                                % zet) 

% % caption maar 0.8  \textwidth breed maken, en normaal lettertype (dit was reeds  zo, maar nu kun je
% % zien wat je kan veranderen indien nodig) (versie 0.7)
% \changecaptionwidth
% \captionwidth{0.8\textwidth}
% %\normalcaptionwidth
% \captiontitlefont{\normalsize} ==> werkt nog niet, later nog eens zien

\usepackage{memhfixc}  % anders krijg je een error (memoir samen met hyperref...)

%%% Local Variables: 
%%% mode: latex
%%% TeX-master: "eindwerk_template"
%%% End: 
